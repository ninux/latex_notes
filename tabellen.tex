\section{Tabellen}

\indent
Tabellen in \LaTeX~zählen sicherlich zu den
Königsdisziplinen, denn meist ist man sich gewohnt sehr
viele Formatierungen zu benutzen (Ausrichtungen, Farben,
Einrückungen etc.). Will man von Anfang an ``komplexe''
Tabellen erstellen (von Hand) so fällt man leicht auf die
Nase oder tut sich schwer, da dies wirklich eine starke
Umgewöhnung ist.

Um das Prinzip einer Tabelle zu erläutern dient das
folgende Snippet.

\begin{center}
\begin{lstlisting}[caption=Einfache Tabelle]{easytable}
% Einfache Tabelle

\begin{tabular}{r l c}
Tag         & Haupttätigkeit        & Stunden \\
Montag      & Erholen vom Weekend   & 8 \\
Dienstag    & arbeiten              & 8 \\
Mittwoch    & arbeiten              & 8 \\
Donnerstag  & Kernel-Updates prüfen & 12 \\
Freitag     & weekend vorbereiten   & 4 \\
Samstag     & feiern                & 24 \\
Sonntag     & schlafen              & 24 \\
\end{tabular}
\end{lstlisting}
\end{center}

% Einfache Tabelle
\fbox{\parbox{0.9\columnwidth}{
\begin{tabular}{r l c}
Tag         & Haupttätigkeit        & Stunden \\
Montag      & Erholen vom Weekend   & 8 \\
Dienstag    & arbeiten              & 8 \\
Mittwoch    & arbeiten              & 8 \\
Donnerstag  & Kernel-Updates prüfen & 12 \\
Freitag     & weekend vorbereiten   & 4 \\
Samstag     & feiern                & 24 \\
Sonntag     & schlafen              & 24 \\
\end{tabular}
}}

\begin{description}
    \item \lstinline|[tabular]| ist eine Tabellenumgebung
die viele Parameter setzt (wie etwa Ausrichtung, Position
etc.).
    \item \lstinline|[{r l l}]| weist \lstinline|tabular|
an, dass eine Tabelle erzeugt werden soll mit drei Spalten
(entsprechend den drei Buchstaben \lstinline|r l l|) und
dass die 1. Spalte rechtsbündig ist (denn der erste
Buchstaben ist \lstinline|r| für right), die 2. soll
linksbündig sein (\lstinline|l| für left) und die 3. Spalte
soll zentriert werden (\lstinline|c| für center).
\end{description}

\noindent
Um Spalten und Zeilen mit Linien zu Trennen, so wie das oft
gewünscht ist bei Tabellen, kann ein \lstinline|\hline|
eingefügt werden um horizontale Linien zu erstellen und ein
\lstinline||| und Spalten voneinander zu trennen.

\begin{center}
\begin{lstlisting}[caption=Tabelle mit
Trennstrichen]{linetable}
% Einfache Tabelle mit Trennstrichen

\begin{tabular}{r | l | c}
Tag         & Haupttätigkeit        & Stunden \\
\hline
Montag      & Erholen vom Weekend   & 8 \\
Dienstag    & arbeiten              & 8 \\
Mittwoch    & arbeiten              & 8 \\
Donnerstag  & Kernel-Updates prüfen & 12 \\
Freitag     & weekend vorbereiten   & 4 \\
Samstag     & feiern                & 24 \\
Sonntag     & schlafen              & 24 \\
\end{tabular}
\end{lstlisting}
\end{center}

% Einfache Tabelle mit Trennstrichen
\fbox{\parbox{0.9\columnwidth}{
\begin{tabular}{r | l | c}
Tag         & Haupttätigkeit        & Stunden \\
\hline
Montag      & Erholen vom Weekend   & 8 \\
Dienstag    & arbeiten              & 8 \\
Mittwoch    & arbeiten              & 8 \\
Donnerstag  & Kernel-Updates prüfen & 12 \\
Freitag     & weekend vorbereiten   & 4 \\
Samstag     & feiern                & 24 \\
Sonntag     & schlafen              & 24 \\
\end{tabular}
}}\\\\

\noindent
Eine professionelle Tabelle beinhaltet aber nebst passenden
Ausrichtungen der Zellen auch eine allgemein passende Form.
So sollte eine gute Tabelle auch eine Indexierung haben.
Mein Vorschlag um gute Tabellen zu erstellen ist wie folgt.

\begin{center}
\begin{lstlisting}[caption=Bessere Tabelle]{goodtable}
% Bessere Tabelle

\begin{table}[htbp]
\fbox{\parbox{0.9\columnwidth}{
\begin{tabular}{| r | l | c |}
\hline
Tag         & Haupttätigkeit        & Stunden \\
\hline
Montag      & Erholen vom Weekend   & 8 \\
\hline
Dienstag    & arbeiten              & 8 \\
\hline
Mittwoch    & arbeiten              & 8 \\
\hline
Donnerstag  & Kernel-Updates prüfen & 12 \\
\hline
Freitag     & weekend vorbereiten   & 4 \\
\hline
Samstag     & feiern                & 24 \\
\hline
Sonntag     & schlafen              & 24 \\
\hline
\end{tabular} }} \newline\newline
\caption{Wochenüberblick}
\label{tab:wochenueberblick}
\end{table}
\end{lstlisting}
\end{center}


\begin{table}[htbp]
\centering
\fbox{\parbox{0.9\columnwidth}{
\begin{tabular}{| r | l | c |}
\hline
Tag         & Haupttätigkeit        & Stunden \\
\hline
Montag      & Erholen vom Weekend   & 8 \\
\hline
Dienstag    & arbeiten              & 8 \\
\hline
Mittwoch    & arbeiten              & 8 \\
\hline
Donnerstag  & Kernel-Updates prüfen & 12 \\
\hline
Freitag     & weekend vorbereiten   & 4 \\
\hline
Samstag     & feiern                & 24 \\
\hline
Sonntag     & schlafen              & 24 \\
\hline
\end{tabular} }} \newline\newline
\caption{Wochenüberblick}
\label{tab:wochenueberblick}
\end{table}\\\\

\noindent
Wird die Tabelle so erzeugt. so ist diese auch indexiert
und somit kann diese im Tabellenverzeichnis aufgelistet
werden. Dies ist oft verlangt bei grösseren Arbeiten.
