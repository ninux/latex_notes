\section{Mathematik}

\indent
Für die Eingabe von Formeln gibt es zwei Möglichkeiten. Für einzelne Formeln 
innerhalb von Text wird die Formel mit dem Zeichen \verb?$? begrenzt. 
\begin{center}
\begin{lstlisting}[caption=Mathe im Text]{dollarmath}
% Mathematik im Text eingebettet

Die Fläche eines Kreises wird mit der Formel $\pi \cdot r^2$ berechnet. 
\end{lstlisting}
\end{center}
Die Fläche eines Kreises wird mit der Formel $\pi \cdot r^2$ berechnet. \\

Für grössere Gleichungen vom Text abgesetzt wird die Formel mit \verb?\[? 
begonnen und mit \verb?\]? beendet. 
\begin{center}
\begin{lstlisting}[caption=Mathe abgesetzt]{dollarmath}
% Abgesetzte Formel

\[ \sum_{K=0}^\infty \frac{x^K}{K!} \]
\end{lstlisting}
\end{center}
\[ \sum_{K=0}^\infty \frac{x^K}{K!} \]

Innerhalb der Mathematik-Umgebung werden Leerzeichen und Zeilenumbrüche ignoriert. 
\begin{center}
\begin{lstlisting}[caption=Mathe abgesetzt]{dollarmath}
\[ e  = 
m \cdot c     ^2 \]
\end{lstlisting}
\end{center}
\[ e  = 
m \cdot c     ^2 \]
