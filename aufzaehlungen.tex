\section{Aufzählungen}

\noindent
Aufzählungen sind in \LaTeX~gegenüber anderen Tools sehr
einfach aufgebaut und verleiten nicht dazu viele
verschiedenen Typen (verschiedene Aufzählungssymbole und
Einrückungen) zu verwenden. 

\subsection{Einfache Aufzählung}

\noindent
Möchte man eine schlichte Aufzählung erstellen so kann dies
mit folgender Eingabe erfolgen:


\begin{lstlisting}[caption=Einfache Aufzählung,
                   label={lst:simplelist}]{Name}
% Einfache Aufzählung

\begin{itemize}
    \item   Das ist die Nummer Eins
    \item   Nummer zwei kommt an dieser Stelle
    \item   Das Letzte kommt zum Schluss
\end{itemize}
\end{lstlisting}

\fbox{\parbox{0.9\columnwidth}{
\begin{itemize}
    \item   Das ist die Nummer Eins
    \item   Nummer zwei kommt an dieser Stelle
    \item   Das Letzte kommt zum Schluss
\end{itemize}
}}

\subsection{Nummerierte Aufzählung}

\noindent
Möchte man eine Aufzählung machen, welche mit
nummerierungen arbeitet so gibt man lediglich einen anderen
Parameter ein.

\begin{lstlisting}[caption=Nummerierte Aufzählung,
                   label={lst:simplelist}]{Name}
% Einfache Aufzählung

\begin{enumerate}
    \item   Das ist die Nummer Eins
    \item   Nummer zwei kommt an dieser Stelle
    \item   Das Letzte kommt zum Schluss
\end{enumerate}
\end{lstlisting}

\fbox{\parbox{0.9\columnwidth}{
\begin{enumerate}
    \item   Das ist die Nummer Eins
    \item   Nummer zwei kommt an dieser Stelle
    \item   Das Letzte kommt zum Schluss
\end{enumerate}
}}

\subsection{Alphabetisierte Aufzählungen}

\indent
Manchmal möchte man Aufzählungen haben, welche alphabetisch
inkrementieren statt numerisch. Dies bedarf eines
zusätzlichen Settings.

\begin{lstlisting}[caption=Alphabetisierte Aufzählung,
                   label={lst:simplelist}]{Name}
% in der Präambel wird folgendes benötigt:
\usepackage{enumitem}

% Im Dokument folgt dann an entsprechender Stelle die
% Alphabetisierte Aufzählung

\begin{enumerate}[label={\alph*)}]
    \item   Das ist die Nummer Eins
    \item   Nummer zwei kommt an dieser Stelle
    \item   Das Letzte kommt zum Schluss
\end{enumerate}

\end{lstlisting}

\fbox{\parbox{0.9\columnwidth}{
\begin{enumerate}[label={\alph*)}]
    \item   Das ist die Nummer Eins
    \item   Nummer zwei kommt an dieser Stelle
    \item   Das Letzte kommt zum Schluss
\end{enumerate}
}}

\subsection{Beschreibende Aufzählung}

\noindent 
Ein exotischeres Aufzählungsdesign ist die
\lstinline|description|. Diese ist je Element nochmals
zweigeteilt in Schlagworte und dazugehörige Beschreibung.
Das Schlagwort wird fett gedruckt und die Beschreibung wird
einfach eingerückt.

\begin{lstlisting}[caption=Alphabetisierte Aufzählung,
                   label={lst:simplelist}]{Name}
% Beschreibende Aufzählung

\begin{description}
    \item [LaTeX] ist ein Softwarepaket, das die Benutzung
          des Textprogramms TeX mit Hilfe von Makros.
    \item [Word] ist ein unfreies aber auch unfaires 
          Textverarbeitungsprogramm der Firma Microsoft.
    \item [AbiWord] ist ein freies, GPL-lizenziertes
          Textverarbeitungsprogramm, das unter Linux und
          Microsoft Windows verfügbar ist.
\end{description}
\end{lstlisting}

\fbox{\parbox{0.9\columnwidth}{
\begin{description}
    \item [LaTeX] ist ein Softwarepaket, das die Benutzung
          des Textprogramms TeX mit Hilfe von Makros.
          Das Programm TeX wurde von Donald E. Knuth,
          Professor an der Stanford-University,
          entwickelt. Leslie Lamport entwickelte Anfang
          der 1980er Jahre[1] darauf aufbauend LaTeX, eine
          Sammlung von TeX-Makros. Der Name ist eine
          Abkürzung für Lamport TeX. Heute ist LaTeX die
          am weitesten verbreitete Methode, TeX zu
          verwenden.
    \item [MS Word] ist ein unfreies aber auch unfaires 
          Textverarbeitungsprogramm der Firma Microsoft.
          Word wurde auf IBM PC-DOS (1983), Apple Macintosh
          (1984), AT\&T Unix (1985), Atari ST (1986), SCO
          UNIX, OS/2, und Microsoft Windows (1989)
          portiert.
    \item [AbiWord] ist ein freies, GPL-lizenziertes
          Textverarbeitungsprogramm, das unter Linux und
          Microsoft Windows verfügbar ist. Für andere
          Betriebssysteme wie Mac OS X, SkyOS und AmigaOS
          4 wird AbiWord nicht weiter gepflegt. Der Name
          „AbiWord“ ist abgeleitet von der Wurzel des
          spanischen Wortes „abierto“ für „offen“.
\end{description}
}}

