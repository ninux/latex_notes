
\noindent
Dokumente sind viel mehr als eine Sammlung von DIN
genormten Seiten aus Papier oder Megabytes an Daten in Ihrem
Filesystem. Sie beherrschen unseren Alltag und unser
Leben mehr als wird uns dies vorstellen können oder wollen.
Egal ob Sie zur Post, Ihrem Vermieter, Ihrem Arbeitgeber,
Ihrer Bank, einer Manpage und was es sonst noch gibt an
Institutionen und Dingen, bei welchem Menschen miteinander
Kommunizieren oder auf andere Wege in Kontakt geraten, sind
Dokumente verschiedener Arten nicht mehr wegzudenken. Sie
geben alle samt (mal mehr mal weniger nützliche)
Informationen wieder und regeln oft Prozesse oder
Tätigkeiten. Oft ist es so, dass Dokumente wie das Wort es
schon beinhaltet, etwas dokumentieren. Es ist gerade dieses
Dokumentieren, welches in aller Regel die höchste
Informationsdichte enthält an Informationen, die ein Leser
sucht. 

Es ist jedem klar, dass gutes dokumentieren viel Zeit
und Erfahrung braucht. Umso bedeutender ist somit, dass man
gerade beim dokumentieren effizient arbeiten sollte. Hier
ist nebst den sprachlichen und analytischen Fähigkeiten des
Schreiber einer Dokumentation auch seine Erfahrung und
Umgang mit Werkzeugen gefragt. Analog zu einem
handwerklichem Beruf kann auch kein Schreiber arbeiten ohne
entsprechende Werkzeuge. Genauso kann auch kein Zimmermann
ohne Hammer und Säge zimmern. Wie bereits erwähnt ist
Erfahrung wie bei vielen anderen Tätigkeiten eine
Grundvoraussetzung und es ist eine allgemeine Tatsache, dass
der Mensch ein sogenanntes Gewohnheitstier ist. Neue
Forschungen zeigen aber dass Kreativität, der
Schlüsselfaktor für die Menschheitsgeschichte und deren
rasante Entwicklung, im Gegensatz zur Erfahrung steht. Man
konnte beweisen, dass man einen Mensch darauf trainieren
kann nicht kreativ zu sein. Dieses Training ist denkbar
einfach; man lässt ihn repetitiv arbeiten. Das heisst, man
erledigt ähnliche Aufgabenstellung mehrmals mit ähnlichen
oder identischen Lösungswegen und ist danach nicht mehr
fähig neue Lösungen zu erarbeiten. Solche Verhaltensmuster
sind umso tragischer, wenn diese bewirken, dass
unpraktische und ineffiziente Lösungen manifestiert sind
obschon seit langem bessere Lösungen bestehen. 

Ohne zu einem Glaubenskrieg aufzurufen möchte an dieser
Stelle der alte Konflikt aufgezeigt werden zwischen Gut und
Böse, die Dunkelheit gegen das Licht, die Konsole gegen das
GUI und wie zu erwarten war, MS-Office gegen \LaTeX.

An dieser Stelle soll betont sein, dass stets die
Philosophie gilt; nutze was dir dient! Dieser Satz soll
seine Gültigkeit auch in dieser Diskussion nicht
verlieren. Jedoch sind sich viele nicht bewusst, dass sie
es sind, die dem dienen, was sie nutzen! 

In diesem Werk wird keine Einführung in \LaTeX~vorgestellt,
sondern es soll lediglich ein alltägliches Nachschlagewerk
abbilden für die gängigen Fragen die der Umgang mit
\LaTeX~aufwerfen kann.