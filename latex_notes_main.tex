% Dokumenteigenschaften bzw. Dokeumentenklasse
\documentclass[a4paper,
               10pt,
               fleqn]{article}

% Texteigenschaften
\usepackage[utf8]{inputenc}    % utf8x kann alle
                                % Textcodierungen
                                % interpretieren
\usepackage[T1]{fontenc}        % Schriftcodierung mit UTF-8
\usepackage{textcomp}           % Erweiterung von fontenc
\usepackage{lmodern}            % Erweiterung des







\usepackage{courier}






\def\LuXeria{{
\rm L\kern-.1667em\kern-.125em\lower-.55ex\hbox{u}\kern-.125emX\kern-.2em\lower-.25ex\hbox{e}\kern-.0em\lower-.5ex\hbox{r}\kern-0.125em\lower.5ex\hbox{i}\kern-0.125\lower-.25ex\hbox{a}
}}








\usepackage{dtklogos}

%\PrerenderUnicode{ä}
%\PrerenderUnicode{ü}
%\PrerenderUnicode{ö}


% Grafikpakete
\usepackage{graphics}
\usepackage{graphicx}

% Spracheigenschaften
\usepackage[ngerman]{babel}     % ngerman = Neues
                                % Deutsch; babel =
                                % internationalisierung
                                % einschalten

% Links im PDF erzeugen (für Verzeichnisse, URLs etc.)
\usepackage{hyperref}

% Mathepakete
\usepackage{amsmath}
\usepackage[all]{xy}

% Glossarpakete
\usepackage[xindy]{glossaries}
\usepackage{makeidx}

% PDF-Pakete
\usepackage{pdfpages}

% Spezielle Grafikpakete
\usepackage{graphicx}

% Source-Code Paket
\definecolor{darkgreen}{rgb}{0,0.6,0}
\usepackage{listings}
\lstset{language=[LaTeX]TeX}
\lstloadlanguages{TeX}
\lstset{basicstyle=\ttfamily,
        numbers=left,
        numberstyle=\tiny, 
        numbersep=5pt,
        breaklines=true,
        texcsstyle=\color{black},
        backgroundcolor=\color{gray!10},
        commentstyle=\color{darkgreen},
        %keywordstyle=\color{red}\bfseries,
        %stringstyle=\color{blue}\bfseries,
        frame=single,
        tabsize=2,
        rulecolor=\color{black!30},
        title=\lstname,
        escapeinside={\%*}{*)},
        breaklines=true,
        breakatwhitespace=true,
        framextopmargin=2pt,
        framexbottommargin=2pt,
        inputencoding=utf8,
        extendedchars=true,
        literate={Ö}{{\"O}}1
                 {Ä}{{\"A}}1
                 {Ü}{{\"U}}1
                 {ü}{{\"u}}1
                 {ä}{{\"a}}1
                 {ö}{{\"o}}1 }
    
    

% ???
\usepackage{printlen}

% Lorem-Ipsum Paket
\usepackage{blindtext}	% generiert sprachlich korrekten "Fülltext"
\usepackage{lipsum}		% generiert klassischen "lorem-ipsum"

% Euro-Betrag Zeichen Paket
\usepackage{eurosym}

% Abkürzungs-Paket 
\usepackage{acronym}

% Auflistungs-Paket (für Auflistungen mit "a)", "b)"...
\usepackage{enumitem}

% URL-Paket (URLs richtig darstellen und umbrechen z.B. in Literaturverzeichnissen
\usepackage{url}

% Zitier-Paket
\usepackage{cite}		% allgemeines Paket
\usepackage{apacite}	% Zitat-Paket für APA-Norm Zitate

% Kopf- und Fusszeilen Paket
\usepackage{fancyhdr}

%%%%%%%%%%%%%%%%%%%%%%%%%%%%%%%%%%%%%%%%%%%%%%%%%%%%%%%%%%%%%%%%%%%%%%%%%%%%%%%%
%%% Kopf und Fusszeilen definieren
%%%%%%%%%%%%%%%%%%%%%%%%%%%%%%%%%%%%%%%%%%%%%%%%%%%%%%%%%%%%%%%%%%%%%%%%%%%%%%%%

\pagestyle{fancy}		% deklaieren dass ein eigener Syle benutzt wird, eben "fancy"
\fancyhf{}				% alle Kopf- und Fusszeilenfelder bereinigen

\addtolength{\textwidth}{1cm}			% anpassen der textbreite
\addtolength{\evensidemargin}{-5mm}		% anpassen des Einzugs für gerade Seiten
\addtolength{\oddsidemargin}{-5mm}		% anpassen des
                                       % Einzugs für
                                       % ungerade Seiten

\renewcommand{\sectionmark}[1]{\markright{#1}{}}


\addtolength{\headwidth}{1cm}			% anpassen der Kopf/Fusszeilenbreite (Summe von den Oberen)

\fancyhead[L]{\LuXeria} 			        % Kopfzeile links
\fancyhead[C]{\LaTeX~Notizen}            % Kopfzeile mitte
\fancyhead[R]{\rightmark}	            % Kopfzeile rechts

\renewcommand{\headrulewidth}{0.4pt} 	% obere Trennlinie

\fancyfoot[L]{Ervin Mazlagic}			% Fusszeile links
\fancyfoot[C]{\today}					% Fusszeile mitte
\fancyfoot[R]{\thepage}					% Fusszeile rechts

\renewcommand{\footrulewidth}{0.4pt} 	%untere Trennlinie

%%%%%%%%%%%%%%%%%%%%%%%%%%%%%%%%%%%%%%%%%%%%%%%%%%%%%%%%%%%%%%%%%%%%%%%%%%%%%%%%
%%% Ende der Präambel
%%%%%%%%%%%%%%%%%%%%%%%%%%%%%%%%%%%%%%%%%%%%%%%%%%%%%%%%%%%%%%%%%%%%%%%%%%%%%%%%

% Anfang des Dokumenteninhaltes
\begin{document}


\begin{titlepage}
\begin{center}
\vfill{\textbf{LuXeria}}
\vfill{\small Ervin Mazlagic}
\vfill{}
\vfill{}
\vfill{}
\vfill{}
\vfill{}
\vfill{\Huge \textbf{\LaTeX~How-To}}
\vfill{}
\vfill{}
\vfill{}
\vfill{}
\vfill{}
\vfill{\textbf{LuXeria - Open Source, Open Mind!}}
\vfill{Adligenswil, \today}
\end{center}
\end{titlepage}


\section*{Einleitung}
\input{einleitung}
\newpage
\tableofcontents
\newpage   
\section{Präambel}
\noindent
Die Präambel ist das Fundament eines \LaTeX~Projektes. Es
ist eben diese Präambel welche viele Einsteiger schockiert
und gerade zu Beginn oder besser gesagt bevor man überhaupt
angefangen hat mit einer Arbeit einen auf die Nase fallen
lässt. Hier wird nun ein setting vorgestellt, welches
allgemeine Bedürfnisse abdeckt.

\subsection{Dokumentenklasse}

Die Dokumentenklasse definiert, was man eigentlich
schreibt. Die hier vorgenommene Einstellung setzt alle
Default-Werte zu der entsprechenden Klasse. Mit Klasse
meint man Bücher, Artikel, Berichte etc.

\begin{center}
\begin{lstlisting}[caption=Dokumentenklasse]{docclass}
% Dokumenteigenschaften bzw. Dokumentenklasse

\documentclass[a4paper,     % Ausgabeformat (A5, A4 etc.)
               10pt,        % Schriftgrösse
               fleqn]       % Formeln ausrichten
               {article}    % Artikel, Bericht, Buch etc.
\end{lstlisting}
\end{center}

\noindent
Wählt man als Klasse beispielsweise \lstinline|article| so
wird alles per Default so eingestellt, dass es am ehesten
einem Artikel\footnote{Artikel bezeichnen bei \LaTeX~in
aller Regel wissenschaftliche Artikel.} entspricht.

\subsection{Texteigenschaften}

\begin{center}
\begin{lstlisting}[caption=Texteigenschaften]{textconf}
% Texteigenschaften

\usepackage[utf8x]{inputenc}    % utf8x kann alle
                                % Textcodierungen
                                % interpretieren
\usepackage[T1]{fontenc}        % Schriftcodierung mit UTF-8
\usepackage{textcomp}           % Erweiterung von fontenc
\usepackage{lmodern}            % Erweiterung des

\PrerenderUnicode{ä}            % PrerenderUnicode bewirkt
\PrerenderUnicode{ü}            % dass Umlaute im PDF 
\PrerenderUnicode{ö}            % korrekt dargestellt werden
\end{lstlisting}
\end{center}

\noindent
Hier ist wichtig zu erwähnen, dass das Codierunssetting vom
Editor bzw. dem System abhängen kann. Bei manchen ist statt
\lstinline|utf8x| die Option \lstinline|uft8| die besser.
Es kann aber auch eine ganz andere sein wie etwa
\lstinline|latin1| (welches bei vielen Windows Usern
Anwendung finden wird).

\subsection{Grafik}

\begin{center}
\begin{lstlisting}[caption=Grafik]{graficconf}
% Grafikpakete

\usepackage{graphics}       % Basis-Grafikpaket (TeX)
\usepackage{graphicx}       % Extended Version
\end{lstlisting}
\end{center}

\subsection{Sprache}

\begin{center}
\begin{lstlisting}[caption=Spracheigenschaften]{sprachconf}
% Spracheigenschaften

\usepackage[english,            % Englisch
            ngerman]            % Neue dt. Rechtschreibung
            {babel}             % internationalisierung
                                % einschalten
\end{lstlisting}
\end{center}

\subsection{HyperLinks}

\begin{center}
\begin{lstlisting}[caption=Dynamische Links]{hyperlinkconf}
% Links im PDF erzeugen (für Verzeichnisse, URLs etc.)

\usepackage{hyperref}
\end{lstlisting}
\end{center}

\noindent
Beim Paket \lstinline|hyperref| sollte man darauf achten,
diese moeglichst zu Beginn in der Praeambel zu verwenden.
Es kann bzw. es kommt zu Problemen wenn es erst nach
bestimmten anderen Paketen geladen wird.

\subsection{Mathematik}

\begin{center}
\begin{lstlisting}[caption=Mathematik]{mathconf}
% Mathepakete

\usepackage{amsmath}
\usepackage[all]{xy}
\DeclareMathOperator{\arccosh}{arccosh}  % Neue Math. Funktion
\end{lstlisting}
\end{center}

\subsection{PDF}

\begin{center}
\begin{lstlisting}[caption=PDF-Paket]{pdfconf}
% PDF-Paket

\usepackage{pdfpages}
\end{lstlisting}
\end{center}

\subsection{SourceCode}

\begin{center}
\begin{lstlisting}[caption=Source-Code Paket]{sourceconf}
% Source-Code Paket

\definecolor{darkgreen}{rgb}{0,0.6,0}
\usepackage{listings} 
\lstset{basicstyle=\ttfamily,
        numbers=left,
        numberstyle=\tiny, 
        numbersep=5pt,
        breaklines=true,
        backgroundcolor=\color{gray!10},
        commentstyle=\color{darkgreen},
        keywordstyle=\color{red},
        frame=single,
        tabsize=2,
        rulecolor=\color{black!30},
        title=\lstname,
        breaklines=true,
        breakatwhitespace=true,
        framextopmargin=2pt,
        framexbottommargin=2pt,
        inputencoding=utf8,
        extendedchars=true,
        literate={Ö}{{\"O}}1
                 {Ä}{{\"A}}1
                 {Ü}{{\"U}}1
                 {ü}{{\"u}}1
                 {ä}{{\"a}}1
                 {ö}{{\"o}}1 }
\lstset{language=Tex}
\lstloadlanguages{TeX}
\end{lstlisting}
\end{center}

\subsection{Fülltext \& Lorem Ipsum}

\begin{center}
\begin{lstlisting}[caption=Fuelltext]{fillconf}
% Fülltexte

\usepackage{blindtext}  % generiert sprachlich korrekten
                        % "Fülltext"
\usepackage{lipsum}     % generiert klassischen
                        % "lorem-ipsum"
\end{lstlisting}
\end{center}

\subsection{Spezielle Symbole}

\begin{center}
\begin{lstlisting}[caption=Euro-Symbol]{symbolconf}
% Euro Symbol

\usepackage{eurosym}
\end{lstlisting}
\end{center}

\subsection{Abkürzungen}

\begin{center}
\begin{lstlisting}[caption=Abkuerzungen]{Name}
% Abkürzungs-Paket 

\usepackage{acronym}
\end{lstlisting}
\end{center}

\subsection{Aufzählungen}

\begin{center}
\begin{lstlisting}[caption=Auflistungen]{Name}
% Auflistungs-Paket (für Auflistungen mit "a)", "b)"...

\usepackage{enumitem}
\end{lstlisting}
\end{center}

\subsection{Literaturverzeichnis}

\begin{center}
\begin{lstlisting}[caption=Auflistungen]{Name}
% URL-Paket (URLs richtig darstellen z.B. in
% Literaturverzeichnissen

\usepackage{url}
\end{lstlisting}
\end{center}

\subsection{Zitieren}

\begin{center}
\begin{lstlisting}[caption=Zitieren]{Name}
% Zitier-Paket

\usepackage{cite}       % allgemeines Paket
\usepackage{apacite}    % Zitat-Paket für APA-Norm Zitate
\usepackage{natbib}     % ein  bekanntes Zitierpaket mit
                        % vielen Optionen und Optimierungen
\end{lstlisting}
\end{center}

\subsection{Kopf- und Fusszeilen}

\begin{center}
\begin{lstlisting}[caption=Kopf- und Fusszeilen]{Name}
% Kopf- und Fusszeilen Paket
\usepackage{fancyhdr}


% Kopf und Fusszeilen definieren:

\pagestyle{fancy}       % deklaieren dass ein eigener Syle
                        % benutzt wird, eben "fancy"
\fancyhf{}              % alle Kopf- und Fusszeilenfelder
                        % bereinigen
% anpassen der Textbreite
\addtolength{\textwidth}{1cm}           

% anpassen des Einzugs für gerade und ungerade Seiten
\addtolength{\evensidemargin}{-5mm}
\addtolength{\oddsidemargin}{-5mm}

\renewcommand{\sectionmark}[1]{\markright{#1}{}}

% anpassen der Kopf/Fusszeilenbreite (Summe von den Oberen)
\addtolength{\headwidth}{1cm}           

\fancyhead[L]{LuXeria}                  % Kopfzeile links
\fancyhead[C]{\LaTeX~Notizen}           % Kopfzeile mitte
\fancyhead[R]{\rightmark}               % Kopfzeile rechts

\renewcommand{\headrulewidth}{0.4pt}    % obere Trennlinie

\fancyfoot[L]{Ervin Mazlagic}           % Fusszeile links
\fancyfoot[C]{\today}                   % Fusszeile mitte
\fancyfoot[R]{\thepage}                 % Fusszeile rechts

\renewcommand{\footrulewidth}{0.4pt}    %untere Trennlinie
\end{lstlisting}
\end{center}





\include{titel}
\include{aufzaehlungen}
\section{Bilder}

\noindent 
Der Umgang in \LaTeX~neigt zu automatisiertem Gebrauch von
bestehenden Lösungen. Dies zeigt sich als Copy-Paste Aktion
bei vielen Usern. Hierzu kann bei entsprechendem Bedarf
eine eigene Code-Completion Sequenz geschriben werden für
die Geeks unter den Lesern.

Ich empfehle wärmstens Bilder immer wie folgt einzufügen:

\begin{lstlisting}[caption=Bilder einfuegen,
                   label={lst:pictures}]{Name}
% Bild einfügen
                   
\begin{figure}[htbp]
    \centering
    \includegraphics[angle=0,
                     width=0.6\textwidth]
                     {meinbild.jpg}
    \caption{Neulich in der Bar}
    \label{pic:inderbar}
\end{figure}
\end{lstlisting}

                  
\begin{figure}[htbp]
% Bild einfügen                
    \centering
\fbox{   
\includegraphics[angle=0,width=0.6\textwidth]{meinbild.jpg}}
    \caption{Neulich in der Bar}
    \label{pic:inderbar}
\end{figure}


\noindent
Zu dem Listing gibt es noch das eine oder anderen zu
erklären:

\begin{description}
    \item \lstinline|[htbp]| Beschreibt wie die Figur
positioniert           wird.
           Jeder der Buchstaben stellt einen Parameter dar.
           Mit der Reihenfolge htbp ergibt das die
           Anweisung; stelle mein Bild bitte hier hin (h
           für here) falls das nicht geht bitte das Bild
           nach oben schieben (t für top) falls das auch
           nicht geht dann eben nach unten (b für bottom)
           und falls das dann auch nicht gehen sollte, dann
           stells doch auf eine neue Seite (p für
           pagebreak).
    \item \lstinline|[angle=0]| Erlaubt das drehen des
Bildes
           (0-360 Grad). Hier mit 0 eingestellt.
    \item \lstinline|[width=0.6\textwidth]| Skaliert das
Bild. Hier
            zum 0.4-fachen der Textbreite. Zu
            beachten ist, dass           Textbreite nicht
            immer der breite des aktuellen          
            Absatzes entspricht. Beispielsweise bei Texten  
            mit mehreren Spalten beschreibt Textbreite die  
            Breite des Textes über die gesamte Seite. Möchte
            man die Bilder der Spalte (Kolonne) angepasst   
            haben so kann statt \lstinline|\textwidth| der  
            Parameter \lstinline|\columnwidth| verwendet    
            werden.
\end{description}



\section{Brief}

\begin{lstlisting}[caption=Allgemeine Settings,
                   label={lst:simplelist}]{Name}
\documentclass[11pt]{g-brief}
\usepackage[utf8]{inputenc}
\usepackage{ngerman}
\usepackage{enumerate}
\usepackage{eurosym}
\lochermarke
\faltmarken
\fenstermarken
\trennlinien                   
\end{lstlisting}

\begin{lstlisting}[caption=Angaben Absender,
                   label={lst:simplelist}]{Name}
% Angaben des Absenders
\Name                {Max Sendermann}
\Strasse             {Senderstrasse 99}
\Zusatz              {}
\RetourAdresse       {}
\Ort                 {6000 Luzern}
\Land                {Schweiz}

\Telefon             {+41/41 450 88 77}
\Telefax             {+41/41 450 88 78}
\Telex               {}
\HTTP                {www.sendermann.ch}
\EMail               {info@sendermann.ch}

\Bank                {Post-Finance}
\BLZ                 {}
\Konto               {40-123456-05}

\Unterschrift        {Max Sendermann}                  
\end{lstlisting}

\begin{lstlisting}[caption=Postvermerk und Empfaenger,
                   label={lst:simplelist}]{Name}
% Sendungsart/Postvermerk
\Postvermerk         {A-Post}

% Angaben des Empfängers
\Adresse             {Moritz Empfängermann\\
                      Empfangsstrasse 11\\
                      8600 Zürich}                  
\end{lstlisting}

\begin{lstlisting}[caption=Betreff und Datum,
                   label={lst:simplelist}]{Name}
% Betreff, Datum und Zeichen

\Betreff             {Empfangsbestätigung}

\Datum               {\today}
\IhrZeichen          {}
\IhrSchreiben        {}
\MeinZeichen         {}
\end{lstlisting}

\begin{lstlisting}[caption=Briefinhalt,
                   label={lst:simplelist}]{Name}
% Anrede & Gruss
\Anrede              {Sehr geehrter Herr Empfängermann,}
\Gruss               {Mit freundlichem Gruss}{1cm}

% Anhang/Anlagen
\Anlagen             {eBay-Kaufbestätigung\\
                      Einzahlungsschein}
\Verteiler           {}

% Briefinhalt
\begin{document}
\begin{g-brief}
Wie per Mail vereinbart, sende ich Ihnen hiermit die
Empfangsbestätigung des von Ihnen auf eBay erworbenen
Artikels. Die von eBay generierte Kaufbestätigung und der
quittierte Einzahlungsschein sind als Kopie hinterlegt zu
diesem Brief als Anlage.
\end{g-brief}
\end{document}

\endinput                 
\end{lstlisting}

\begin{figure}[htbp]
% Bild einfügen                
    \centering
\fbox{   
\includegraphics[width=1.0\textwidth]{briefm_muster.pdf} }
    \caption{Brief}
    \label{pic:inderbar}
\end{figure}
\section{Tabellen}

\indent
Tabellen in \LaTeX~zählen sicherlich zu den
Königsdisziplinen, denn meist ist man sich gewohnt sehr
viele Formatierungen zu benutzen (Ausrichtungen, Farben,
Einrückungen etc.). Will man von Anfang an ``komplexe''
Tabellen erstellen (von Hand) so fällt man leicht auf die
Nase oder tut sich schwer, da dies wirklich eine starke
Umgewöhnung ist.

Um das Prinzip einer Tabelle zu erläutern dient das
folgende Snippet.

\begin{center}
\begin{lstlisting}[caption=Einfache Tabelle]{easytable}
% Einfache Tabelle

\begin{tabular}{r l c}
Tag         & Haupttätigkeit        & Stunden \\
Montag      & Erholen vom Weekend   & 8 \\
Dienstag    & arbeiten              & 8 \\
Mittwoch    & arbeiten              & 8 \\
Donnerstag  & Kernel-Updates prüfen & 12 \\
Freitag     & weekend vorbereiten   & 4 \\
Samstag     & feiern                & 24 \\
Sonntag     & schlafen              & 24 \\
\end{tabular}
\end{lstlisting}
\end{center}

% Einfache Tabelle
\fbox{\parbox{0.9\columnwidth}{
\begin{tabular}{r l c}
Tag         & Haupttätigkeit        & Stunden \\
Montag      & Erholen vom Weekend   & 8 \\
Dienstag    & arbeiten              & 8 \\
Mittwoch    & arbeiten              & 8 \\
Donnerstag  & Kernel-Updates prüfen & 12 \\
Freitag     & weekend vorbereiten   & 4 \\
Samstag     & feiern                & 24 \\
Sonntag     & schlafen              & 24 \\
\end{tabular}
}}

\begin{description}
    \item \lstinline|[tabular]| ist eine Tabellenumgebung
die viele Parameter setzt (wie etwa Ausrichtung, Position
etc.).
    \item \lstinline|[{r l l}]| weist \lstinline|tabular|
an, dass eine Tabelle erzeugt werden soll mit drei Spalten
(entsprechend den drei Buchstaben \lstinline|r l l|) und
dass die 1. Spalte rechtsbündig ist (denn der erste
Buchstaben ist \lstinline|r| für right), die 2. soll
linksbündig sein (\lstinline|l| für left) und die 3. Spalte
soll zentriert werden (\lstinline|c| für center).
\end{description}

\noindent
Um Spalten und Zeilen mit Linien zu Trennen, so wie das oft
gewünscht ist bei Tabellen, kann ein \lstinline|\hline|
eingefügt werden um horizontale Linien zu erstellen und ein
\lstinline||| und Spalten voneinander zu trennen.

\begin{center}
\begin{lstlisting}[caption=Tabelle mit
Trennstrichen]{linetable}
% Einfache Tabelle mit Trennstrichen

\begin{tabular}{r | l | c}
Tag         & Haupttätigkeit        & Stunden \\
\hline
Montag      & Erholen vom Weekend   & 8 \\
Dienstag    & arbeiten              & 8 \\
Mittwoch    & arbeiten              & 8 \\
Donnerstag  & Kernel-Updates prüfen & 12 \\
Freitag     & weekend vorbereiten   & 4 \\
Samstag     & feiern                & 24 \\
Sonntag     & schlafen              & 24 \\
\end{tabular}
\end{lstlisting}
\end{center}

% Einfache Tabelle mit Trennstrichen
\fbox{\parbox{0.9\columnwidth}{
\begin{tabular}{r | l | c}
Tag         & Haupttätigkeit        & Stunden \\
\hline
Montag      & Erholen vom Weekend   & 8 \\
Dienstag    & arbeiten              & 8 \\
Mittwoch    & arbeiten              & 8 \\
Donnerstag  & Kernel-Updates prüfen & 12 \\
Freitag     & weekend vorbereiten   & 4 \\
Samstag     & feiern                & 24 \\
Sonntag     & schlafen              & 24 \\
\end{tabular}
}}\\\\

\noindent
Eine professionelle Tabelle beinhaltet aber nebst passenden
Ausrichtungen der Zellen auch eine allgemein passende Form.
So sollte eine gute Tabelle auch eine Indexierung haben.
Mein Vorschlag um gute Tabellen zu erstellen ist wie folgt.

\begin{center}
\begin{lstlisting}[caption=Bessere Tabelle]{goodtable}
% Bessere Tabelle

\begin{table}[htbp]
\fbox{\parbox{0.9\columnwidth}{
\begin{tabular}{| r | l | c |}
\hline
Tag         & Haupttätigkeit        & Stunden \\
\hline
Montag      & Erholen vom Weekend   & 8 \\
\hline
Dienstag    & arbeiten              & 8 \\
\hline
Mittwoch    & arbeiten              & 8 \\
\hline
Donnerstag  & Kernel-Updates prüfen & 12 \\
\hline
Freitag     & weekend vorbereiten   & 4 \\
\hline
Samstag     & feiern                & 24 \\
\hline
Sonntag     & schlafen              & 24 \\
\hline
\end{tabular} }} \newline\newline
\caption{Wochenüberblick}
\label{tab:wochenueberblick}
\end{table}
\end{lstlisting}
\end{center}


\begin{table}[htbp]
\centering
\fbox{\parbox{0.9\columnwidth}{
\begin{tabular}{| r | l | c |}
\hline
Tag         & Haupttätigkeit        & Stunden \\
\hline
Montag      & Erholen vom Weekend   & 8 \\
\hline
Dienstag    & arbeiten              & 8 \\
\hline
Mittwoch    & arbeiten              & 8 \\
\hline
Donnerstag  & Kernel-Updates prüfen & 12 \\
\hline
Freitag     & weekend vorbereiten   & 4 \\
\hline
Samstag     & feiern                & 24 \\
\hline
Sonntag     & schlafen              & 24 \\
\hline
\end{tabular} }} \newline\newline
\caption{Wochenüberblick}
\label{tab:wochenueberblick}
\end{table}

\noindent
Wird die Tabelle so erzeugt. so ist diese auch indexiert
und somit kann diese im Tabellenverzeichnis aufgelistet
werden. Dies ist oft verlangt bei grösseren Arbeiten.

% \section{Mathematik}

\indent
Für die Eingabe von Formeln gibt es zwei Möglichkeiten. Für einzelne Formeln 
innerhalb von Text wird die Formel mit dem Zeichen \verb?$? begrenzt. 
\begin{center}
\begin{lstlisting}[caption=Mathe im Text]{dollarmath}
% Mathematik im Text eingebettet

Die Fläche eines Kreises wird mit der Formel $\pi \cdot r^2$ berechnet. 
\end{lstlisting}
\end{center}
Die Fläche eines Kreises wird mit der Formel $\pi \cdot r^2$ berechnet. \\

Für grössere Gleichungen vom Text abgesetzt wird die Formel mit \verb?\[? 
begonnen und mit \verb?\]? beendet. 
\begin{center}
\begin{lstlisting}[caption=Mathe abgesetzt]{dollarmath}
% Abgesetzte Formel

\[ \sum_{K=0}^\infty \frac{x^K}{K!} \]
\end{lstlisting}
\end{center}
\[ \sum_{K=0}^\infty \frac{x^K}{K!} \]

Innerhalb der Mathematik-Umgebung werden Leerzeichen und Zeilenumbrüche ignoriert. 
\begin{center}
\begin{lstlisting}[caption=Mathe abgesetzt]{dollarmath}
\[ e  = 
m \cdot c     ^2 \]
\end{lstlisting}
\end{center}
\[ e  = 
m \cdot c     ^2 \]

\section{Literaturverzeichnis}

\noindent
Literaturverzeichnisse sind eine sehr wichtige Komponente
in vielen Arbeiten und man sollte diese verwenden.

Ein solches in \LaTeX~zu erstellen ist relativ einfach. Man
erstellt ein separates File welches die Quellen beinhaltet.
Dieses wird dann Zusammen mit dem Tool \BibTeX~genutzt zur
Erzeugung der Verzeichnisse. 

Viele Verlagseiten und auch z.B. die Wikipedia bietet einen automatischen
Export für \BibTeX~Einträge. Hier ein Beispiel zum Wikipedia-Artikel zu \LaTeX. \cite{wiki:latex}

\begin{center}
\begin{lstlisting}[caption=BibTeX Export aus Wikipedia]{wikibib}
@misc{ wiki:latex,
   author = "Wikipedia",
   title  = "LaTeX --- Wikipedia{,} Die freie Enzyklopädie",
   year   = "2012",
   url    = "http://de.wikipedia.org/w/index.
             php?title=LaTeX&oldid=110063618",
   note   = "[Online; Stand 2. Dezember 2012]"
 }
\end{lstlisting}
\end{center}

\noindent
Möchte man diese Quelle in seinem Dokument zitieren so gibt es je nach Style 
des Zitierens ein anderes Kommando. Bei der in diesem Werk empfohlenem APA Style
genügt es einfach \lstinline|\cite{wiki:latex}| zu schreiben.

\begin{description}
    \item \lstinline|[@misc]| weist \BibTeX~an, dass es sich bei dieser Quelle
    vom typ misc (Abkürzung von miscellaneous, also ``diverses'') handelt.
    Falls man ein Buch zitier möchte so gibt man stattdessen \lstinline|@book| 
    ein etc.
    \item \lstinline|[wiki:latex]| ist das Makro für diese Quelle. D.h. man 
    schreibt im Dokument eben dieses Makro um auf diese Quelle zu referenzieren.
    \item \lstinline|["Wikipedia"]| Statt die Gänsefüsschem zu schreiben kann man
    auch einfach geschweifte Klammern verwenden \lstinline|{Wikipedia}|.
    Hier ist noch nützlich zu wissen, dass doppelte Geschweifte Klammern  die 
    Inhalte innerhalb der Geschweiften Klammern nicht verändert in ihrer 
    Gross- Kleinschreibung. Dies gilt aber nicht immer. Oft geht es auch ohne.
\end{description}

\noindent 
Hier folgt nun ein Beispiel von einem Bib File mit verschiedenen Quellen.

\begin{center}
\begin{lstlisting}[caption=BibTeX Datei mit verschiedenen Quellen]{masterbib}
@misc{ wiki:latex,
   author = "Wikipedia",
   title  = "{LaTeX --- Wikipedia{,} Die freie Enzyklopädie}",
   year   = "2012",
   url    = "http://de.wikipedia.org/w/index.
             php?title=LaTeX&oldid=110063618",
   note   = "[Online; Stand 2. Dezember 2012]"
 }
 
@Book{kompakt_latex,
  author     = "Reinhard Wonneberger",
  title      = "Kompaktfuehrer LaTe{X}",
  address    = "Bonn",
  year       = "1988",
  descriptor = "LaTeX, TeX",
}

@Book{lamport_latex,
  author    = "Leslie Lamport",
  title     = "{{\LaTeX}}",
  publisher = "Cyfronet",
  address   = "Krak{\'o}w",
  pages     = "x + 202",
  year      = "1991",
  bibdate   = "Wed Jun 22 18:19:42 MDT 2005",
  bibsource = "alpha.bn.org.pl:210/INNOPAC;
               http://www.math.utah.edu/pub/
               tex/bib/texbook3.bib",
  note      =  "Polish translation of ``{\LaTeX}: 
                A Document Preparation System'', 
                1986, by Piotr Wyrostek.",
  acknowledgement = "Nelson H. F. Beebe, University 
         of Utah, Department
         of Mathematics, 110 LCB, 155 S 1400 E RM 
         233, Salt Lake City, UT 84112-0090, USA, 
         Tel: +1 801 581 5254, FAX: +1
         801 581 4148, e-mail: \path|beebe@math.utah.edu|,
         \path|beebe@acm.org|, \path|beebe@computer.org|
         (Internet), URL:
         \path|http://www.math.utah.edu/~beebe/|",
  language =    "Polish",
  subject  = "LATEX; podr{\k{e}}cznik",
}

\end{lstlisting}
\end{center}

\noindent
Die im Dokument zitierten Werke werden bei der Compilation\footnote{Immer 
zuerst \BibTeX ausführen und dann \LaTeX, da \LaTeX~die Ergebnisse aus \BibTeX~
einbindet. Es ist eine typische \LaTeX~Gewohnheit die Compilationen jeweils 
mehrfach durchzuführen, da verschachtelungen oft nicht rekursiv auf einmal 
ausgeführt werden können.} in das Literaturverzeichnis eingefügt.

Möchte man alle Quellen aus dem \BibTeX~File aufgelistet haben, unabhängig 
davon ob diese zitiert wurden so kann man \lstinline|\nocite{*}| verwenden.




% Abbildungsverzeichnis einfügen
\listoffigures

% Tabellenverzeichnis einfügen
\listoftables

% Listing-Verzeichnis
\lstlistoflistings

% Abkürzungsverzeichnis einfügen
% >>> Die Liste der Abkürzungen ist in einem separaten File
\include{meine_abkuerzungen}			

% Literaturverzeichnis einfügen
\nocite{*}								% alle Quellen auflisten
\bibliography{literatur}				    % Bibliographie wählen (BibTex Datei)
\bibliographystyle{apacite}				% Zitationsstyl wählen (hier nach APA-Norm)

% Ende des Dokumenteninhaltes
\end{document}
