\section{Bilder}

\noindent 
Der Umgang in \LaTeX~neigt zu automatisiertem Gebrauch von
bestehenden Lösungen. Dies zeigt sich als Copy-Paste Aktion
bei vielen Usern. Hierzu kann bei entsprechendem Bedarf
eine eigene Code-Completion Sequenz geschriben werden für
die Geeks unter den Lesern.

Ich empfehle wärmstens Bilder immer wie folgt einzufügen:

\begin{lstlisting}[caption=Bilder einfuegen,
                   label={lst:pictures}]{Name}
% Bild einfügen
                   
\begin{figure}[htbp]
    \centering
    \includegraphics[angle=0,
                     width=0.6\textwidth]
                     {meinbild.jpg}
    \caption{Neulich in der Bar}
    \label{pic:inderbar}
\end{figure}
\end{lstlisting}

                  
\begin{figure}[htbp]
% Bild einfügen                
    \centering
\fbox{   
\includegraphics[angle=0,width=0.6\textwidth]{meinbild.jpg}}
    \caption{Neulich in der Bar}
    \label{pic:inderbar}
\end{figure}


\noindent
Zu dem Listing gibt es noch das eine oder anderen zu
erklären:

\begin{description}
    \item \lstinline|[htbp]| Beschreibt wie die Figur
positioniert           wird.
           Jeder der Buchstaben stellt einen Parameter dar.
           Mit der Reihenfolge htbp ergibt das die
           Anweisung; stelle mein Bild bitte hier hin (h
           für here) falls das nicht geht bitte das Bild
           nach oben schieben (t für top) falls das auch
           nicht geht dann eben nach unten (b für bottom)
           und falls das dann auch nicht gehen sollte, dann
           stells doch auf eine neue Seite (p für
           pagebreak).
    \item \lstinline|[angle=0]| Erlaubt das drehen des
Bildes
           (0-360 Grad). Hier mit 0 eingestellt.
    \item \lstinline|[width=0.6\textwidth]| Skaliert das
Bild. Hier
            zum 0.4-fachen der Textbreite. Zu
            beachten ist, dass           Textbreite nicht
            immer der breite des aktuellen          
            Absatzes entspricht. Beispielsweise bei Texten  
            mit mehreren Spalten beschreibt Textbreite die  
            Breite des Textes über die gesamte Seite. Möchte
            man die Bilder der Spalte (Kolonne) angepasst   
            haben so kann statt \lstinline|\textwidth| der  
            Parameter \lstinline|\columnwidth| verwendet    
            werden.
\end{description}


