\section{Literaturverzeichnis}

\noindent
Literaturverzeichnisse sind eine sehr wichtige Komponente
in vielen Arbeiten und man sollte diese verwenden.

Ein solches in \LaTeX~zu erstellen ist relativ einfach. Man
erstellt ein separates File welches die Quellen beinhaltet.
Dieses wird dann Zusammen mit dem Tool \BibTeX~genutzt zur
Erzeugung der Verzeichnisse. 

Viele Verlagseiten und auch z.B. die Wikipedia bietet einen automatischen
Export für \BibTeX~Einträge. Hier ein Beispiel zum Wikipedia-Artikel zu \LaTeX. \cite{wiki:latex}

\begin{center}
\begin{lstlisting}[caption=BibTeX Export aus Wikipedia]{wikibib}
@misc{ wiki:latex,
   author = "Wikipedia",
   title  = "LaTeX --- Wikipedia{,} Die freie Enzyklopädie",
   year   = "2012",
   url    = "http://de.wikipedia.org/w/index.
             php?title=LaTeX&oldid=110063618",
   note   = "[Online; Stand 2. Dezember 2012]"
 }
\end{lstlisting}
\end{center}

\noindent
Möchte man diese Quelle in seinem Dokument zitieren so gibt es je nach Style 
des Zitierens ein anderes Kommando. Bei der in diesem Werk empfohlenem APA Style
genügt es einfach \lstinline|\cite{wiki:latex}| zu schreiben.

\begin{description}
    \item \lstinline|[@misc]| weist \BibTeX~an, dass es sich bei dieser Quelle
    vom typ misc (Abkürzung von miscellaneous, also ``diverses'') handelt.
    Falls man ein Buch zitier möchte so gibt man stattdessen \lstinline|@book| 
    ein etc.
    \item \lstinline|[wiki:latex]| ist das Makro für diese Quelle. D.h. man 
    schreibt im Dokument eben dieses Makro um auf diese Quelle zu referenzieren.
    \item \lstinline|["Wikipedia"]| Statt die Gänsefüsschem zu schreiben kann man
    auch einfach geschweifte Klammern verwenden \lstinline|{Wikipedia}|.
    Hier ist noch nützlich zu wissen, dass doppelte Geschweifte Klammern  die 
    Inhalte innerhalb der Geschweiften Klammern nicht verändert in ihrer 
    Gross- Kleinschreibung. Dies gilt aber nicht immer. Oft geht es auch ohne.
\end{description}

\noindent 
Hier folgt nun ein Beispiel von einem Bib File mit verschiedenen Quellen.

\begin{center}
\begin{lstlisting}[caption=BibTeX Datei mit verschiedenen Quellen]{masterbib}
@misc{ wiki:latex,
   author = "Wikipedia",
   title  = "{LaTeX --- Wikipedia{,} Die freie Enzyklopädie}",
   year   = "2012",
   url    = "http://de.wikipedia.org/w/index.
             php?title=LaTeX&oldid=110063618",
   note   = "[Online; Stand 2. Dezember 2012]"
 }
 
@Book{kompakt_latex,
  author     = "Reinhard Wonneberger",
  title      = "Kompaktfuehrer LaTe{X}",
  address    = "Bonn",
  year       = "1988",
  descriptor = "LaTeX, TeX",
}

@Book{lamport_latex,
  author    = "Leslie Lamport",
  title     = "{{\LaTeX}}",
  publisher = "Cyfronet",
  address   = "Krak{\'o}w",
  pages     = "x + 202",
  year      = "1991",
  bibdate   = "Wed Jun 22 18:19:42 MDT 2005",
  bibsource = "alpha.bn.org.pl:210/INNOPAC;
               http://www.math.utah.edu/pub/
               tex/bib/texbook3.bib",
  note      =  "Polish translation of ``{\LaTeX}: 
                A Document Preparation System'', 
                1986, by Piotr Wyrostek.",
  acknowledgement = "Nelson H. F. Beebe, University 
         of Utah, Department
         of Mathematics, 110 LCB, 155 S 1400 E RM 
         233, Salt Lake City, UT 84112-0090, USA, 
         Tel: +1 801 581 5254, FAX: +1
         801 581 4148, e-mail: \path|beebe@math.utah.edu|,
         \path|beebe@acm.org|, \path|beebe@computer.org|
         (Internet), URL:
         \path|http://www.math.utah.edu/~beebe/|",
  language =    "Polish",
  subject  = "LATEX; podr{\k{e}}cznik",
}

\end{lstlisting}
\end{center}

\noindent
Die im Dokument zitierten Werke werden bei der Compilation\footnote{Immer 
zuerst \BibTeX ausführen und dann \LaTeX, da \LaTeX~die Ergebnisse aus \BibTeX~
einbindet. Es ist eine typische \LaTeX~Gewohnheit die Compilationen jeweils 
mehrfach durchzuführen, da verschachtelungen oft nicht rekursiv auf einmal 
ausgeführt werden können.} in das Literaturverzeichnis eingefügt.

Möchte man alle Quellen aus dem \BibTeX~File aufgelistet haben, unabhängig 
davon ob diese zitiert wurden so kann man \lstinline|\nocite{*}| verwenden.


