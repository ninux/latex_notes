\section{Präambel}
\noindent
Die Präambel ist das Fundament eines \LaTeX~Projektes. Es
ist eben diese Präambel welche viele Einsteiger schockiert
und gerade zu Beginn oder besser gesagt bevor man überhaupt
angefangen hat mit einer Arbeit einen auf die Nase fallen
lässt. Hier wird nun ein setting vorgestellt, welches
allgemeine Bedürfnisse abdeckt.

\subsection{Dokumentenklasse}

Die Dokumentenklasse definiert, was man eigentlich
schreibt. Die hier vorgenommene Einstellung setzt alle
Default-Werte zu der entsprechenden Klasse. Mit Klasse
meint man Bücher, Artikel, Berichte etc.

\begin{center}
\begin{lstlisting}[caption=Dokumentenklasse]{docclass}
% Dokumenteigenschaften bzw. Dokumentenklasse

\documentclass[a4paper,     % Ausgabeformat (A5, A4 etc.)
               10pt,        % Schriftgrösse
               fleqn]       % Formeln ausrichten
               {article}    % Artikel, Bericht, Buch etc.
\end{lstlisting}
\end{center}

\noindent
Wählt man als Klasse beispielsweise \lstinline|article| so
wird alles per Default so eingestellt, dass es am ehesten
einem Artikel\footnote{Artikel bezeichnen bei \LaTeX~in
aller Regel wissenschaftliche Artikel.} entspricht.

\subsection{Texteigenschaften}

\begin{center}
\begin{lstlisting}[caption=Texteigenschaften]{textconf}
% Texteigenschaften

\usepackage[utf8x]{inputenc}    % utf8x kann alle
                                % Textcodierungen
                                % interpretieren
\usepackage[T1]{fontenc}        % Schriftcodierung mit UTF-8
\usepackage{textcomp}           % Erweiterung von fontenc
\usepackage{lmodern}            % Erweiterung des

\PrerenderUnicode{ä}            % PrerenderUnicode bewirkt
\PrerenderUnicode{ü}            % dass Umlaute im PDF 
\PrerenderUnicode{ö}            % korrekt dargestellt werden
\end{lstlisting}
\end{center}

\noindent
Hier ist wichtig zu erwähnen, dass das Codierunssetting vom
Editor bzw. dem System abhängen kann. Bei manchen ist statt
\lstinline|utf8x| die Option \lstinline|uft8| die besser.
Es kann aber auch eine ganz andere sein wie etwa
\lstinline|latin1| (welches bei vielen Windows Usern
Anwendung finden wird).

\subsection{Grafik}

\begin{center}
\begin{lstlisting}[caption=Grafik]{graficconf}
% Grafikpakete

\usepackage{graphics}       % Basis-Grafikpaket (TeX)
\usepackage{graphicx}       % Extended Version
\end{lstlisting}
\end{center}

\subsection{Sprache}

\begin{center}
\begin{lstlisting}[caption=Spracheigenschaften]{sprachconf}
% Spracheigenschaften

\usepackage[english,            % Englisch
            ngerman]            % Neue dt. Rechtschreibung
            {babel}             % internationalisierung
                                % einschalten
\end{lstlisting}
\end{center}

\subsection{HyperLinks}

\begin{center}
\begin{lstlisting}[caption=Dynamische Links]{hyperlinkconf}
% Links im PDF erzeugen (für Verzeichnisse, URLs etc.)

\usepackage{hyperref}
\end{lstlisting}
\end{center}

\noindent
Beim Paket \lstinline|hyperref| sollte man darauf achten,
diese moeglichst zu Beginn in der Praeambel zu verwenden.
Es kann bzw. es kommt zu Problemen wenn es erst nach
bestimmten anderen Paketen geladen wird.

\subsection{Mathematik}

\begin{center}
\begin{lstlisting}[caption=Mathematik]{mathconf}
% Mathepakete

\usepackage{amsmath}
\usepackage[all]{xy}
\DeclareMathOperator{\arccosh}{arccosh}  % Neue Math. Funktion
\end{lstlisting}
\end{center}

\subsection{PDF}

\begin{center}
\begin{lstlisting}[caption=PDF-Paket]{pdfconf}
% PDF-Paket

\usepackage{pdfpages}
\end{lstlisting}
\end{center}

\subsection{SourceCode}

\begin{center}
\begin{lstlisting}[caption=Source-Code Paket]{sourceconf}
% Source-Code Paket

\definecolor{darkgreen}{rgb}{0,0.6,0}
\usepackage{listings} 
\lstset{basicstyle=\ttfamily,
        numbers=left,
        numberstyle=\tiny, 
        numbersep=5pt,
        breaklines=true,
        backgroundcolor=\color{gray!10},
        commentstyle=\color{darkgreen},
        keywordstyle=\color{red},
        frame=single,
        tabsize=2,
        rulecolor=\color{black!30},
        title=\lstname,
        breaklines=true,
        breakatwhitespace=true,
        framextopmargin=2pt,
        framexbottommargin=2pt,
        inputencoding=utf8,
        extendedchars=true,
        literate={Ö}{{\"O}}1
                 {Ä}{{\"A}}1
                 {Ü}{{\"U}}1
                 {ü}{{\"u}}1
                 {ä}{{\"a}}1
                 {ö}{{\"o}}1 }
\lstset{language=Tex}
\lstloadlanguages{TeX}
\end{lstlisting}
\end{center}

\subsection{Fülltext \& Lorem Ipsum}

\begin{center}
\begin{lstlisting}[caption=Fuelltext]{fillconf}
% Fülltexte

\usepackage{blindtext}  % generiert sprachlich korrekten
                        % "Fülltext"
\usepackage{lipsum}     % generiert klassischen
                        % "lorem-ipsum"
\end{lstlisting}
\end{center}

\subsection{Spezielle Symbole}

\begin{center}
\begin{lstlisting}[caption=Euro-Symbol]{symbolconf}
% Euro Symbol

\usepackage{eurosym}
\end{lstlisting}
\end{center}

\subsection{Abkürzungen}

\begin{center}
\begin{lstlisting}[caption=Abkuerzungen]{Name}
% Abkürzungs-Paket 

\usepackage{acronym}
\end{lstlisting}
\end{center}

\subsection{Aufzählungen}

\begin{center}
\begin{lstlisting}[caption=Auflistungen]{Name}
% Auflistungs-Paket (für Auflistungen mit "a)", "b)"...

\usepackage{enumitem}
\end{lstlisting}
\end{center}

\subsection{Literaturverzeichnis}

\begin{center}
\begin{lstlisting}[caption=Auflistungen]{Name}
% URL-Paket (URLs richtig darstellen z.B. in
% Literaturverzeichnissen

\usepackage{url}
\end{lstlisting}
\end{center}

\subsection{Zitieren}

\begin{center}
\begin{lstlisting}[caption=Zitieren]{Name}
% Zitier-Paket

\usepackage{cite}       % allgemeines Paket
\usepackage{apacite}    % Zitat-Paket für APA-Norm Zitate
\usepackage{natbib}     % ein  bekanntes Zitierpaket mit
                        % vielen Optionen und Optimierungen
\end{lstlisting}
\end{center}

\subsection{Kopf- und Fusszeilen}

\begin{center}
\begin{lstlisting}[caption=Kopf- und Fusszeilen]{Name}
% Kopf- und Fusszeilen Paket
\usepackage{fancyhdr}


% Kopf und Fusszeilen definieren:

\pagestyle{fancy}       % deklaieren dass ein eigener Syle
                        % benutzt wird, eben "fancy"
\fancyhf{}              % alle Kopf- und Fusszeilenfelder
                        % bereinigen
% anpassen der Textbreite
\addtolength{\textwidth}{1cm}           

% anpassen des Einzugs für gerade und ungerade Seiten
\addtolength{\evensidemargin}{-5mm}
\addtolength{\oddsidemargin}{-5mm}

\renewcommand{\sectionmark}[1]{\markright{#1}{}}

% anpassen der Kopf/Fusszeilenbreite (Summe von den Oberen)
\addtolength{\headwidth}{1cm}           

\fancyhead[L]{LuXeria}                  % Kopfzeile links
\fancyhead[C]{\LaTeX~Notizen}           % Kopfzeile mitte
\fancyhead[R]{\rightmark}               % Kopfzeile rechts

\renewcommand{\headrulewidth}{0.4pt}    % obere Trennlinie

\fancyfoot[L]{Ervin Mazlagic}           % Fusszeile links
\fancyfoot[C]{\today}                   % Fusszeile mitte
\fancyfoot[R]{\thepage}                 % Fusszeile rechts

\renewcommand{\footrulewidth}{0.4pt}    %untere Trennlinie
\end{lstlisting}
\end{center}




